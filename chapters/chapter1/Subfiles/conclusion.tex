\onlyinsubfile{\setcounter{section}{7}}
\section{Conclusion}
\notinsubfile{\label{sec:Conclusion}}

\par This paper explores the ability of macroeconomic models to generate a distribution of wealth with substantial inequality by the estimation of a calibrated consumption-saving model that allows for heterogeneous returns. Consistent with other deviations from the representative agent framework, I find that the ex-ante heterogeneity in rates of return needed to match wealth moments compares well with empirical estimates of the returns to net worth found by recent work.   

\par Unlike much of the existing literature linking persistent return heterogeneity to factors such as entrepreneurial ability or financial sophistication, I focus more on heterogeneity in  deposit rates across banks. I incorporate related literature in the standard HA framework under a simple but realistic assumption that many households remain \say{stuck} with the bank in or around their neighborhood. That bank makes complex financial decisions that ultimately affect households through the channel of varying deposit rates offered.

\par Although I do not allow households to switch banks, this mechanism is similar to financial literacy explanations for returns heterogeneity, without the need to model portfolio risk. This exclusion is useful because (i) untangling how much of the persistent component of returns comes from risk preferences and from financial sophistication is not straightforward and (ii) there is significant heterogeneity in returns even when individuals hold no risky assets.

\par This model is a partial equilibrium analysis. The market interest rate is being taken as given. It is not determined by some market clearing condition. I view my model as the simplest implementation of a potential source of heterogeneity. In the simulation of the model and the resulting SMM estimation, I do not add banks as an agent type. Thus, the bank is not responding in every period to the level of deposits they receive after they set the optimal deposit rate based on the demand for deposits. Consequently, I avoid having to choose a particular scheme of allocating agents in the model to a particular bank. In this way, there are seven types of banks just as there are seven types of returns that an agent may receive.

\par The culmination of these simplfiying assumptions leaves us with a setting where we can consider the most stark role for returns in explaining wealth inequality. With more features in the model, like general equilibrium considerations, endogenous returns heterogeneity, portfolio choice, overlapping generations and bequests, etc., we leave more room for these features to explain the generated skewness in the model's distribution of wealth. 

% \subsubsection{Equilibrium deposit rates and bank owners}

% \par After proposing a profit-maximization problem for banks, a natural question is: where do the profits of the banks go? 

% \par The answer can be seen in the following scenario. Consider the case where there are only two bank, one globally integrated and one local. The globally integrated bank has an elasticity of deposit rates which is very elastic, and the local bank has an elastcity which is extremely inelastic. This suggests that the former will need to offer a deposit rate which is close to the prevailing world interest rate, else they will face a large reduction in their foreign deposit holdings. The latter will offer a deposit rate which is significantly lower than the world interest rate since they do not face this issue.

% \par In this scenario, the globally integrated bank will earn a profit that can be normalized to 1, and the local bank will earn a profit that can be normalized to a value that is strictly greater than 1 in each period. This implies that the stream of revenue for the entire period (or infinity horizon) for the local bank is strictly higher than the stream of revenue associated with the globally integrated bank. Furthermore, from the literature on asset pricing, we know that a setting such as this will result in a share price for owning the local bank which is strictly larger than the share price for owning the globally integrated bank, so that the ratio of revenue stream to share price is equal for the two bank types.

\par

\par